% !TEX TS-program = xelatex
\documentclass[12pt,a4paper,oneside]{abntex2}

\usepackage[brazil]{babel}
\usepackage{fontspec}
\usepackage{geometry}
\usepackage{graphicx}
\usepackage{fancyhdr}
\usepackage{setspace}
\usepackage{xcolor}
\usepackage{hyperref}
\usepackage{titlesec}
\usepackage{enumitem}
\usepackage{eso-pic}
\usepackage{tikzpagenodes}
\usepackage{ragged2e}
\AtBeginDocument{\let\tt\ttfamily}

\geometry{left=3cm,right=2cm,top=3cm,bottom=2cm}

\definecolor{organizepurple}{HTML}{681582}
\definecolor{organizegreen}{HTML}{62F3BC}
\definecolor{textgray}{HTML}{1A1A1A}

\IfFileExists{fonts/Open_Sans/OpenSans-Regular.ttf}{
  \setmainfont{OpenSans}[
    Path=fonts/Open_Sans/,
    Extension=.ttf,
    UprightFont=*-Regular,
    ItalicFont=*-Italic,
    BoldFont=*-Semibold,
    BoldItalicFont=*-SemiboldItalic,
    Ligatures=TeX
  ]
}{
  \IfFontExistsTF{Open Sans}{
    \setmainfont{Open Sans}[Ligatures=TeX]
  }{
    \setmainfont{Arial}[Ligatures=TeX]
  }
}

\newfontfamily\OrganizeHeading[
  Path=fonts/League_Spartan/static/,
  Extension=.ttf,
  UprightFont=*-SemiBold,
  BoldFont=*-Black,
  Ligatures=TeX
]{LeagueSpartan}

\newfontfamily\OrganizeAlt[
  Path=fonts/Lexend_Deca/static/,
  Extension=.ttf,
  UprightFont=*-Regular,
  BoldFont=*-SemiBold,
  Ligatures=TeX
]{LexendDeca}

\OnehalfSpacing
\setlength{\parindent}{1.25cm}
\setlength{\parskip}{0pt}
\raggedbottom
\justifying

\setcounter{secnumdepth}{3}
\setcounter{tocdepth}{3}
\titlespacing*{\section}{0pt}{1.5\baselineskip}{1.5\baselineskip}
\titlespacing*{\subsection}{0pt}{1.5\baselineskip}{1.5\baselineskip}
\titlespacing*{\subsubsection}{0pt}{1.5\baselineskip}{1.5\baselineskip}
\setsecheadstyle{\OrganizeHeading\normalsize\bfseries}
\setsubsecheadstyle{\OrganizeHeading\normalsize\bfseries}
\setsubsubsecheadstyle{\OrganizeAlt\normalsize}
\renewcommand{\thesection}{\arabic{section}}
\renewcommand{\thesubsection}{\thesection.\arabic{subsection}}
\renewcommand{\thesubsubsection}{\thesubsection.\arabic{subsubsection}}
\setboolean{ABNTEXupperchapter}{true}
\setboolean{ABNTEXuppersection}{true}
\setboolean{ABNTEXuppersubsection}{false}
\titleformat{\section}
  {\OrganizeHeading\normalsize\bfseries\color{textgray}}
  {\thesection}{0.5em}{\MakeUppercase}
\titleformat{\subsection}
  {\OrganizeHeading\normalsize\bfseries\color{textgray}}
  {\thesubsection}{0.5em}{}
\titleformat{\subsubsection}
  {\OrganizeAlt\normalsize\color{textgray}}
  {\thesubsubsection}{0.5em}{}

% Resumo/Abstract com título em estilo de seção (League Spartan 12pt, caixa alta, centralizado para pré-textuais)
\newenvironment{organizeresumo}[1][\resumoname]{%
  \SingleSpacing
  \par\noindent{\centering\OrganizeHeading\normalsize\bfseries\MakeUppercase{#1}\par}\vspace{1.5\baselineskip}%
}{\par\vspace{1.0\baselineskip}}

\renewcommand{\resumoname}{RESUMO}
\renewcommand{\abstractname}{ABSTRACT}
\renewcommand{\contentsname}{SUMÁRIO}
\renewcommand{\printtoctitle}[1]{%
  \begin{center}
    {\normalsize\bfseries\MakeUppercase{#1}}\par
  \end{center}%
}
\renewcommand{\cftsectionfont}{\normalfont\bfseries\color{textgray}}
\renewcommand{\cftsectionpagefont}{\normalfont\bfseries\color{textgray}}
\renewcommand{\cftsubsectionfont}{\normalfont\bfseries\color{textgray}}
\renewcommand{\cftsubsectionpagefont}{\normalfont\bfseries\color{textgray}}
\renewcommand{\cftsubsubsectionfont}{\normalfont\color{textgray}}
\renewcommand{\cftsubsubsectionpagefont}{\normalfont\color{textgray}}
\setlength{\cftsectionindent}{0pt}
\setlength{\cftsectionnumwidth}{1em}
\setlength{\cftsubsectionindent}{0pt}
\setlength{\cftsubsectionnumwidth}{1.6em}

\renewenvironment{citacao}{%
  \list{}{%
    \listparindent=\parindent
    \parsep=0pt plus 1pt
    \leftmargin=4cm
    \rightmargin=0cm
  }%
  \item\relax\fontsize{10}{15}\selectfont
}{\endlist}

\newlength{\organizebandheight}
\setlength{\organizebandheight}{2.6cm}

\newcommand{\organizerodape}{%
  \parbox{\textwidth}{\centering
    \OrganizeAlt\fontsize{9}{11}\selectfont\color{textgray}%
    R. Silveira Martins, 2555 – Departamento de Educação (DEDC I), Cabula, Salvador – BA, CEP 41150-000\\%
    \color{organizepurple}@organizejr \textbullet\ presidencia@organizejr.com}%
}

\setlength{\headheight}{15pt}
\setlength{\headsep}{1cm}
\setlength{\footskip}{1.2cm}

\fancypagestyle{organize}{%
  \fancyhf{}%
  \fancyhead[R]{\OrganizeAlt\fontsize{10}{12}\selectfont\textcolor{white}{\thepage}}
  \fancyfoot[C]{\organizerodape}%
  \renewcommand{\headrulewidth}{0pt}%
  \renewcommand{\footrulewidth}{0pt}%
}
\fancypagestyle{organizepre}{%
  \fancyhf{}%
  \fancyfoot[C]{\organizerodape}%
  \renewcommand{\headrulewidth}{0pt}%
  \renewcommand{\footrulewidth}{0pt}%
}
\fancypagestyle{plain}{%
  \fancyhf{}%
  \fancyhead[R]{\OrganizeAlt\fontsize{10}{12}\selectfont\thepage}
  \fancyfoot[C]{\organizerodape}%
  \renewcommand{\headrulewidth}{0pt}%
  \renewcommand{\footrulewidth}{0pt}%
}
\pagestyle{organizepre}
\aliaspagestyle{plain}{organizepre}
\aliaspagestyle{chapter}{organizepre}
\aliaspagestyle{simple}{organizepre}
\aliaspagestyle{title}{organizepre}
\aliaspagestyle{titlepage}{organizepre}
\aliaspagestyle{abntheadings}{organizepre}
\aliaspagestyle{empty}{organizepre}

\AddToShipoutPictureBG{%
  \begin{tikzpicture}[remember picture,overlay]
    \fill[organizepurple] (current page.north west) rectangle ([yshift=-\organizebandheight]current page.north east);
    \node at ([yshift=-0.5\organizebandheight]current page.north) {\includegraphics[height=1.18cm]{logo_organizejr.png}};
  \end{tikzpicture}%
}

\hypersetup{
  pdftitle={Papel timbrado Organize Jr.},
  pdfauthor={Organize Jr.},
  pdfsubject={Modelo de timbrado em XeLaTeX + abnTeX2},
  colorlinks=true,
  linkcolor=black,
  urlcolor=organizepurple
}

\newcommand{\nomeinstituicao}{Universidade do Estado da Bahia}
\newcommand{\nomedepartamento}{Departamento de Educação}
\newcommand{\nomecurso}{Empresa Júnior de Psicologia Organizacional da UNEB}
\newcommand{\nomeautor}{[NOME]}
\newcommand{\nomecidade}{Salvador}
\newcommand{\anoacademico}{2025}
\newcommand{\tituloprincipal}{[TITULO]}
\newcommand{\tipodocumento}{Ofício}
\newcommand{\nomedisciplina}{[DISCIPLINA]}
\newcommand{\nomeorientador}{[ORIENTADOR]}

\titulo{\OrganizeHeading\bfseries\normalsize\MakeUppercase{\tituloprincipal}}
\autor{\OrganizeHeading\bfseries\normalsize\MakeUppercase{\nomeautor}}
\instituicao{\OrganizeHeading\normalsize\MakeUppercase{\nomeinstituicao}\\\OrganizeHeading\normalsize\MakeUppercase{\nomedepartamento}\\\OrganizeHeading\normalsize\MakeUppercase{\nomecurso}}
\tipotrabalho{\OrganizeHeading\normalsize\MakeUppercase{\tipodocumento}}
\local{\OrganizeHeading\normalsize\nomecidade}
\data{\OrganizeHeading\normalsize\anoacademico}

\begin{document}
\justifying
\pagestyle{organizepre}

\pretextual
% Capa
\begin{capa}
  \pagestyle{organizepre}\thispagestyle{organizepre}
  \begin{center}
    {\OrganizeHeading\normalsize\MakeUppercase{\nomeinstituicao}\par}
    {\OrganizeHeading\normalsize\MakeUppercase{\nomedepartamento}\par}
    {\OrganizeHeading\normalsize\MakeUppercase{\nomecurso}\par}
    \vfill
    {\OrganizeHeading\normalsize\MakeUppercase{\nomeautor}\par}
    \vspace{1.5cm}
    {\OrganizeHeading\normalsize\MakeUppercase{\tituloprincipal}\par}
    \vfill
    {\OrganizeHeading\normalsize\nomecidade\par}
    {\OrganizeHeading\normalsize\anoacademico\par}
  \end{center}
\end{capa}
\clearpage\pagestyle{organize}\thispagestyle{organize}

% Folha de rosto
\begin{folhaderosto}
  \pagestyle{organizepre}\thispagestyle{organizepre}
  \pagenumbering{arabic}\setcounter{page}{1}
  \SingleSpacing
  {\centering\OrganizeHeading\normalsize\MakeUppercase{\nomeautor}\par}
  \vspace{\fill}
  {\centering\OrganizeHeading\normalsize\MakeUppercase{\tituloprincipal}\par}
  \vspace{1.5cm}
  {
    \SingleSpacing
    \leftskip=7cm
    \parindent=0pt
    \noindent Trabalho apresentado como requisito parcial na disciplina \nomedisciplina, sob orientação de \nomeorientador, no \nomecurso\ da \nomeinstituicao.\par
  }
  \vspace{\fill}
  {\centering\OrganizeHeading\normalsize\nomecidade\par}
  {\centering\OrganizeHeading\normalsize\anoacademico\par}
\end{folhaderosto}

\pagestyle{organizepre}
\aliaspagestyle{plain}{organizepre}

% Resumo e Abstract
\clearpage
\begin{organizeresumo}
\thispagestyle{organizepre}
Texto do resumo sem recuo.\\
\textbf{Palavras-chave:} termo 1; termo 2; termo 3.
\end{organizeresumo}

\clearpage
\begin{organizeresumo}[ABSTRACT]
\thispagestyle{organizepre}
\begin{otherlanguage*}{english}
Abstract text without indentation.\\
\textbf{Keywords:} term 1; term 2; term 3.
\end{otherlanguage*}
\end{organizeresumo}
\clearpage

\pdfbookmark[0]{\contentsname}{toc}
{\pagestyle{organizepre}\thispagestyle{organizepre}%
 \begingroup\begin{spacing}{1.5}\tableofcontents*\end{spacing}\endgroup}
\cleardoublepage

\textual
\pagestyle{organize}
\aliaspagestyle{plain}{organize}
\aliaspagestyle{empty}{organize}
\thispagestyle{organize}

\section{Introdução}
Conteúdo introdutório em Open Sans 12 pt, espaçamento 1,5, recuo 1,25 cm.

\section{Diretrizes de uso}
\begin{itemize}[leftmargin=1.25cm]
  \item Compile com `latexmk -xelatex` para preservar as fontes OpenType.
  \item Altere margens ou espaçamentos em `\\geometry` conforme a necessidade do documento.
  \item O rodapé e o cabeçalho podem ser removidos editando os comandos `\\organizecabecalho` e `\\organizerodape`.
\end{itemize}

\section{Conclusão}
Síntese final.

\postextual

\renewcommand{\bibname}{REFERÊNCIAS}
\renewcommand{\bibsection}{%
  \begin{center}\OrganizeHeading\bfseries\MakeUppercase{\bibname}\end{center}%
  \vspace{0.6cm}%
  \addcontentsline{toc}{section}{\bibname}%
}

\bibliographystyle{abntex2-alf-local}
\begin{filecontents*}{\jobname.bib}
@book{exemplo,
  author = {Autor, Fulano},
  title = {Livro de Exemplo},
  year = {2020},
  publisher = {Editora X},
  address = {São Paulo}
}
\end{filecontents*}
\bibliography{\jobname}

\end{document}
