\begin{filecontents*}{\jobname.bib}
@book{silva2020,
  author    = {Silva, Ana},
  title     = {Metodologia da Pesquisa},
  year      = {2020},
  publisher = {Editora Acad{\^e}mica},
  address   = {S{\~a}o Paulo}
}
@article{pereira2022,
  author  = {Pereira, Jo{\~a}o},
  title   = {Inova{\c c}{\~a}o em Organiza{\c c}{\~o}es},
  journal = {Revista Brasileira de Gest{\~a}o},
  year    = {2022},
  volume  = {18},
  number  = {2},
  pages   = {45--59}
}
\end{filecontents*}

\documentclass[12pt,a4paper,oneside]{abntex2}

\usepackage[brazil]{babel}
\usepackage{fontspec}
\usepackage{geometry}
\usepackage{graphicx}
\usepackage{hyperref}
\usepackage[alf,abnt-emphasize=bf]{abntex2cite}
\usepackage{etoolbox}
\usepackage{xparse}
\usepackage{chngcntr}
\usepackage{titlesec}
\usepackage{enumitem}
\usepackage{xcolor}
\usepackage{fancyhdr}
\usepackage{eso-pic}
\usepackage{tikzpagenodes}
\usepackage{ragged2e}
\usepackage{indentfirst}
\usepackage{lipsum}
\AtBeginDocument{\let\tt\ttfamily}

% Margens ABNT (texto) — timbrado cobre toda a p{\'a}gina no overlay
\geometry{a4paper,left=3cm,right=2cm,top=3cm,bottom=2cm}

\definecolor{organizepurple}{HTML}{681582}
\definecolor{organizegreen}{HTML}{62F3BC}
\definecolor{textgray}{HTML}{1A1A1A}

\IfFileExists{fonts/Open_Sans/OpenSans-Regular.ttf}{
  \setmainfont{OpenSans}[Path=fonts/Open_Sans/,Extension=.ttf,UprightFont=*-Regular,ItalicFont=*-Italic,BoldFont=*-Semibold,BoldItalicFont=*-SemiboldItalic,Ligatures=TeX]
}{
  \IfFontExistsTF{Open Sans}{\setmainfont{Open Sans}[Ligatures=TeX]}{\PackageError{fonts}{Open Sans nao encontrada}{Instale Open Sans ou forneca em fonts/Open_Sans}}
}
\newfontfamily\OrganizeHeading[Path=fonts/League_Spartan/static/,Extension=.ttf,UprightFont=*-Bold,BoldFont=*-Black,Ligatures=TeX]{LeagueSpartan}
\newfontfamily\OrganizeAlt[Path=fonts/Lexend_Deca/static/,Extension=.ttf,UprightFont=*-Regular,BoldFont=*-SemiBold,Ligatures=TeX]{LexendDeca}

\OnehalfSpacing
\setlength{\parindent}{1.25cm}
\setlength{\parskip}{0pt}
\setlength{\footskip}{40pt}
\raggedbottom

\setcounter{secnumdepth}{3}
\setcounter{tocdepth}{3}
\titlespacing*{\section}{0pt}{1.5\baselineskip}{1.5\baselineskip}
\titlespacing*{\subsection}{0pt}{1.5\baselineskip}{1.5\baselineskip}
\titlespacing*{\subsubsection}{0pt}{1.5\baselineskip}{1.5\baselineskip}
\cftsetindents{chapter}{0em}{1em}
\cftsetindents{section}{0em}{2em}
\cftsetindents{subsection}{0em}{2.5em}
\cftsetindents{subsubsection}{0em}{3em}
\setlength{\cftlastnumwidth}{1em}
\renewcommand{\cftchapterfont}{\OrganizeHeading\bfseries\MakeUppercase}
\renewcommand{\cftchapterpagefont}{\OrganizeHeading\bfseries}
\renewcommand{\cftsectionfont}{\OrganizeHeading\bfseries}
\renewcommand{\cftsectionpagefont}{\OrganizeHeading\bfseries}
\renewcommand{\cftsubsectionfont}{\OrganizeAlt\normalfont}
\renewcommand{\cftsubsectionpagefont}{\OrganizeAlt\normalfont}
\setlength{\cftbeforechapterskip}{0pt}
\setlength{\cftbeforesectionskip}{0pt}
\setlength{\cftbeforesubsectionskip}{0pt}
\renewcommand{\ABNTEXchapterfont}{\OrganizeHeading\normalsize\bfseries}
\renewcommand{\ABNTEXchapterfontsize}{\normalsize}
\renewcommand{\ABNTEXsectionfont}{\OrganizeHeading\normalsize\bfseries}
\renewcommand{\ABNTEXsectionfontsize}{\normalsize}
\renewcommand{\ABNTEXsubsectionfont}{\OrganizeAlt\normalsize}
\renewcommand{\ABNTEXsubsectionfontsize}{\normalsize}
\renewcommand{\ABNTEXsubsubsectionfont}{\OrganizeAlt\normalsize}
\renewcommand{\ABNTEXsubsubsectionfontsize}{\normalsize}
\setsecheadstyle{\OrganizeHeading\normalsize\bfseries}
\setsubsecheadstyle{\OrganizeAlt\normalsize\bfseries}
\setsubsubsecheadstyle{\OrganizeAlt\normalsize}
\providecommand{\pretextualfont}{}
\providecommand{\pretextualchapterfont}{}
\renewcommand{\pretextualfont}{\OrganizeHeading\normalsize\bfseries}
\renewcommand{\pretextualchapterfont}{\OrganizeHeading\normalsize\bfseries}
\setboolean{ABNTEXupperchapter}{true}
\setboolean{ABNTEXuppersection}{false}

% Quebra para cap{\'\i}tulos e pr{\'e}-textuais
\pretocmd{\tableofcontents}{\clearpage}{}{}
\AtBeginEnvironment{resumo}{\clearpage}
\AtBeginDocument{\pretocmd{\bibliography}{\cleardoublepage\phantomsection}{}{}}
\preto{\chapter}{\clearpage}

% Alias de cita{\c c}{\~a}o
\NewDocumentCommand{\autorcite}{s O{} m}{%
  \IfBooleanTF{#1}%
    {\hyperlink{cite.#3}{\citeauthoronline{#3} (\citeyear{#3}\IfBlankTF{#2}{}{, p.~#2})}}%
    {\hyperlink{cite.#3}{(\citeauthoronline{#3}, \citeyear{#3}\IfBlankTF{#2}{}{, p.~#2})}}%
}

% Timbrado Organize Jr.
\newlength{\organizebandheight}
\setlength{\organizebandheight}{2.6cm}
\newcommand{\organizecabecalho}{%
  \begin{tikzpicture}[remember picture,overlay]
    \fill[organizepurple] (current page.north west) rectangle ([yshift=-\organizebandheight]current page.north east);
    \node at ([yshift=-0.5\organizebandheight]current page.north) {\includegraphics[height=1.18cm]{logo_organizejr.png}};
  \end{tikzpicture}%
}
\AddToShipoutPictureBG{\organizecabecalho}

\newcommand{\organizerodape}{%
  {\centering\OrganizeAlt\fontsize{9}{11}\selectfont\color{textgray}%
  R. Silveira Martins, 2555 – Departamento de Educa{\c c}{\~a}o (DEDC I), Cabula, Salvador – BA, CEP 41150-000\\%
  \color{organizepurple}@organizejr \textbullet\ presidencia@organizejr.com\par}%
}

\setlength{\headheight}{15pt}
\setlength{\headsep}{1cm}
\setlength{\footskip}{1.2cm}

\fancypagestyle{organize}{%
  \fancyhf{}%
  \fancyhead[R]{\OrganizeAlt\fontsize{10}{12}\selectfont\textcolor{white}{\thepage}}
  \fancyfoot[C]{\organizerodape}%
  \renewcommand{\headrulewidth}{0pt}%
  \renewcommand{\footrulewidth}{0pt}%
}
\fancypagestyle{organizepre}{%
  \fancyhf{}%
  \fancyfoot[C]{\organizerodape}%
  \renewcommand{\headrulewidth}{0pt}%
  \renewcommand{\footrulewidth}{0pt}%
}
\pagestyle{organize}
\aliaspagestyle{plain}{organize}
\aliaspagestyle{chapter}{organize}
\aliaspagestyle{empty}{organize}
\aliaspagestyle{title}{organize}
\aliaspagestyle{titlepage}{organize}
\aliaspagestyle{simple}{organize}
\aliaspagestyle{abntheadings}{organize}

\hypersetup{colorlinks=true,linkcolor=organizepurple,urlcolor=organizepurple}

% Metadados
\newcommand{\nomeinstituicao}{Universidade do Estado da Bahia}
\newcommand{\nomedepartamento}{Departamento de Educa{\c c}{\~a}o}
\newcommand{\nomecurso}{Organize Jr.}
\newcommand{\nomeautor}{Lucas Camilo Carvalho \\ Maria Silva Souza}
\newcommand{\nomecidade}{Salvador}
\newcommand{\anoacademico}{2025}
\newcommand{\tituloprincipal}{Relat{\'o}rio de Teste com Lorem Ipsum}
\newcommand{\titulocomplemento}{Subt{\'\i}tulo de Exemplo}
\newcommand{\tipodocumento}{Relat{\'o}rio}

\begin{document}
% Capa
\begin{capa}
  \pagestyle{organizepre}\thispagestyle{organizepre}
  \begin{center}
    {\OrganizeHeading\normalsize\MakeUppercase{\nomeinstituicao}\par}
    {\OrganizeHeading\normalsize\MakeUppercase{\nomedepartamento}\par}
    {\OrganizeHeading\normalsize\MakeUppercase{\nomecurso}\par}
    \vfill
    {\OrganizeHeading\normalsize\MakeUppercase{\nomeautor}\par}
    \vspace{1.5cm}
            {\OrganizeHeading\normalsize\MakeUppercase{\tituloprincipal}: {\normalfont\normalsize\MakeUppercase{\titulocomplemento}}\par}    \vfill
    {\OrganizeHeading\normalsize\nomecidade\par}
    {\OrganizeHeading\normalsize\anoacademico\par}
  \end{center}
\end{capa}
\clearpage

% Folha de rosto
\begin{folhaderosto}
  \pagestyle{organizepre}\thispagestyle{organizepre}
  \pagenumbering{arabic}\setcounter{page}{1}
  \SingleSpacing
  {\centering\OrganizeHeading\normalsize\MakeUppercase{\nomeautor}\par}
  \vspace{\fill}
  {\centering\OrganizeHeading\normalsize\MakeUppercase{\tituloprincipal}: {\normalfont\normalsize\MakeUppercase{\titulocomplemento}}\par}
  \vspace{1.5cm}
  {
    \SingleSpacing
    \leftskip=7cm
    \parindent=0pt
    \noindent Relat{\'o}rio apresentado {\`a} Organize Jr. como requisito parcial.\par
  }
  \vspace{\fill}
  {\centering\OrganizeHeading\normalsize\nomecidade\par}
  {\centering\OrganizeHeading\normalsize\anoacademico\par}
\end{folhaderosto}
\clearpage

\begin{resumo}
\pagestyle{organize}
Texto de resumo sem recuo, espa{\c c}amento simples. \lipsum[1]\par
\textbf{Palavras-chave:} Organize. Teste. Lorem-ipsum.
\end{resumo}

\begin{resumo}[ABSTRACT]
\begin{otherlanguage*}{english}
Abstract text without indentation. \lipsum[2]\par
\textbf{Keywords:} Organize. Test. Lorem ipsum.
\end{otherlanguage*}
\end{resumo}

\pdfbookmark[0]{\contentsname}{toc}
\tableofcontents*
\cleardoublepage

\renewcommand{\bibname}{\MakeUppercase{REFER\^ENCIAS}}
\renewcommand{\refname}{\MakeUppercase{REFER\^ENCIAS}}

\chapter{INTRODU{\c C}{\~A}O}
\lipsum[3-4] \autorcite{silva2020}

\chapter{DESENVOLVIMENTO}
\section{Revis{\~a}o}
Conforme \autorcite*{pereira2022}, a literatura recente destaca tend{\^e}ncias que precisam ser acompanhadas. \lipsum[5]

\section{Discuss{\~a}o}
\lipsum[6]
\subsection{Exemplo de subse{\c c}{\~a}o terci{\'a}ria}
\lipsum[8]

\begin{citacao}
O planejamento do texto acad{\^e}mico exige organizar ideias em blocos coerentes, mantendo unidade de sentido e clareza argumentativa ao longo de par{\'a}grafos extensos, mesmo quando a exposi{\c c}{\~a}o ultrapassa quatro linhas e demanda cuidado na pontua{\c c}{\~a}o e nas refer{\^e}ncias \autorcite[45]{silva2020}.
\end{citacao}

\chapter{CONCLUS{\~A}O}
\lipsum[7]

\renewcommand{\bibname}{\MakeUppercase{REFER\^ENCIAS}}
\renewcommand{\bibsection}{%
  \addtocontents{toc}{\protect\setlength{\protect\cftchapterindent}{0pt}}%
  \addtocontents{toc}{\protect\setlength{\protect\cftchapternumwidth}{\cftlastnumwidth}}%
  \chapter*{\bibname}%
  \addcontentsline{toc}{chapter}{\bibname}%
}

\bibliographystyle{abntex2-alf-local}
\bibliography{\jobname}
\end{document}
